\begin{abstract}
News reflect intent of media while content in social networks are in majority opinions from the public. Change of topics coverage of the two, thus topic dynamics, can help us to understand the interaction between media and the public. Take Ferguson case as an example, we propose a Single Topic LDA (\stlda) to discover topics of news and tweets under a unified frame, so that topics in tweets and news are comparable. Though trained together, each tweet is supposed to have one topic while news is a distribution of multiple topics. The model performs better in the evaluation of topic quality of tweets and detection of topic dynamics in news and tweets, because \stlda removes noisy topics that conventional LDA assign to tweets. Using the results by \stlda, we compare topic dynamics of tweets in and out of \stlouis area, and the difference and relation between topic coverage of news and tweets. Results show the difference of tweets in and out of \stlouis area, and that news have mainstream topics to cover, while topics in tweets are more diverse and change along with evolvement of events. It partial support the cascade theory that media play a role but the public are not heavily influenced.
\end{abstract} 
