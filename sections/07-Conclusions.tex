\section{Conclusions}
\label{sec:conclu}

We propose a new topic model \stlda to discover topics in news and tweets under a unified frame. Different from conventional LDA, \stlda assigns multiple topics to news while gives only one topic to each tweet. Compared to the results of LDA, topic analysis of tweets shows that the topic given by \stlda can usually accurately summarize the topic of tweets, while LDA usually gives much more irrelevant topics. We also examine the topic dynamics of news and tweets based on results of LDA and \stlda, finding that more diverse topic changes in news can be discovered by \stlda. Moreover, topic dynamics in tweets discovered by \stlda has patterns consistent with real events. It turns out that \stlda has advantage in accurate identification of one topic for each tweet, thus avoiding the representation of noisy topics, which is usually the case in conventional LDA on tweets.

Using topics discovered by \stlda, we compare topic dynamics of tweets in and out of \stlouis area, finding that people in \stlouis area publish more about what happened in the area, and their topics are highly related with event evolvement in Ferguson. However people outside \stlouis area discuss more about the concepts such as \racism, which are abstract perception of Ferguson unrest event. Tweets in and out of Ferguson area share similar topics in \newsreport, \curfew, \shootincident and \michaelbrown.

Comparison of topic dynamics of tweets and news shows that topics in tweets are more diverse than topics in news. News and tweets both talk about the \shootincident and \racism, however \stlda identifies them as separate topics for tweets and news. From the top words in topics we find that though they describe the same event, they are using different vocabulary, which means there is much difference in how media and public talk. On the other hand, media and the public share some common topics such as \protest and \newsreport, but it is hard to identify the direction of influence considering the complicated changes of topics.

In the next step, we want to further find the relation between tweets and news, thus linking them together according to users' behaviors such as retweeting and reply. Social network is another factor that may have influence on how people think and talk about things. Incorporating social links between users is another direction for further understanding why there is the dynamics of topics.

\psrcomment{Overall nice work, and I like the way you laid out the issues and potential contributions in Section 1-2. I don't feel like you achieved the full scope of what you were aiming for, in terms of the question in introduction, but this was reasonable as an initial foray and the qualitative discussion is promising. I think the modeling using a \emph{single} topic space is what's most important and your model does seem to allow both short and long texts to co-exist with the same topics. I don't understand, though, how the full model works, since you seem only to have described the model for tweets in Section 3.}

\psrcomment{An idea to consider: rather than a categorical distribution between tweets (one topic) and news (a distribution), what about a model that selects the ``concentration" of topics based on the document length?}
