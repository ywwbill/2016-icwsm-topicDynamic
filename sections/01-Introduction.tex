\section{Introduction}
\label{sec:intro}

Cascading activation model is a widely accepted model, which explores the relationship between government, media and the public~\cite{entman1993framing}. The model helps to explain how frame of information extends down from the White House to elites, media and then to the public of the system. It proposes that moving downward along the cascade is easy, while spreading ideas from lower levels to upper requires extra energy because of the strength of power, the independent environment for decision and comprehensiveness. As the cascade goes down with framing of work from upper layers, the information becomes limited to highlights. The model assumes a certain direction of information flow. Although it acknowledges variation and the possible ways that news feeds back information about the public to influence actions of higher levels, the stair structure emphasizes heavily on the influence from media to the public. However the emergence of social networks makes us think that the situation has changed. 

The creation of social networks aims to provide everyone with equal access to the world, to encourage more independent role in expression, and to achieve information democracy. It has led to increasing participation of information spread, opinion expression and activism by providing easier information access to diverse roles~\cite{kelly2006protest,gonzalez2011dynamics,tufekci2012social}. Social networks become a new competition battlefield for different frames, because everyone can share, comment and create any information. Broader penetration of certain frames becomes more difficult. Politicians even utilize astroturfing, smear campaigns to influence users' choices~\cite{ratkiewicz2010detecting}. On the other side, influence that comes from the public may gain power because they are able to participate and even organize activities online. During the Arab Spring, Twitter has promoted the protest mobilization through reporting real-time event and provided basis for collaboration and emotional mobilization~\cite{breuer2014social}.

It leads us to rethink about the cascade model, specifically on the blurring role of media and users. As a special function role in society, media is still distinctive, but information spreading between media and the public is in both directions, so we want to study whether the power of influence is blurred for media and the public. The following two questions are emphasized:

\begin{enumerate}
\item Is the public still heavily influenced by frames of the media?
\item Does the media pay attention to what the public concerns and reflect the voice from the public?
\end{enumerate}

The influence is hard to detect, considering the difficulty of capturing the status of the public before and after media news. However, we can study the topics of users and media and how these topics are related in time series, as an evidence of media coverage of the public opinions, or influence of media on users.  

In this study, we take Ferguson unrest event in 2014 as an example, to analyze news and tweets topics along with the evolvement of the case. Based on latent Dirichlet allocation~\cite[LDA]{blei2003latent}, Single Topic LDA (\stlda) is proposed to bring long news documents and short tweets under a unified topic frame, so that topics of news and tweets are comparable. From intuition, we assume that long news documents are comprised with multiple topics, while each tweet has only one topic because of the length limitation. Then daily distribution of topics of news and tweets are compared. We also analyze the difference of topics in geography, hypothesizing that geographical culture congruency may lead to different focuses of the public and thus various reactions to media topics.

The contribution of this study lays in three main aspects. First, we tackle the technical problem of building topic model for a mixture of short and long documents. Conventional topic models such as LDA and PLSA~\cite{hofmann1999probabilistic} perform badly on because co-occurrence patterns in short texts are sparse. Consider a tweet with 5 words, every word's topic assignment makes up 20\% of its topic proportion, but usually a tweet's main topic is decided by 2 or 3 words. In this case, the main topic proportion is about 40\% to 60\%, which does not reflect the truth. Our model considers the words in a tweet as a whole and assigns only one topic to a tweet, so that the main topic is more likely to be assigned.
Second, this preliminary study offers a method to understand which are strong frames, which has been a challenge for social scientists~\cite{chong2007framing}. By examining the topic dynamics between news and social media, which are separately representative of media and the public, we may understand which news is influential or which frames are more competitive by analyzing opinions of the public.
Third, bringing tweets and news to the same topic frames will help us to understand large volumes of tweets through news, thus in a more understandable way. When training topic models separately on news and tweets, it is not easy to understand how topics in news and tweets match each other. When combining tweets and news together for analysis, on the other hand, the latent meaning of a topic is the same for tweets and news. Thus large amount of tweets can be represented by news with similar topic distribution, which makes it easier to know public opinions and estimate for opinion poll. 
