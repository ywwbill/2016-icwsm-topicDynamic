\section{Conclusions}
\label{sec:conclu}

To examine interactions between news and media, we propose a new topic model \stlda to bring news and tweets under a unified topic frame, so that topics of news and tweets are comparable. Meanwhile, \stlda shows advantage in processing documents of different lengths, by assigning multiple topics to news and only one topic to each tweet. It avoids representation of noisy topics in tweets, which is usually the case in conventional LDA on tweets.

To large extent, the results support the theory of cascade model~\cite{entman1993framing} from the perspective of the role media play. News have certain topics to cover. Specifically, a certain space of news is occupied by report of the government reaction. The change of major topics is not directly related with corresponding topics in tweets. Even in minor topics, the evidence of news coverage of tweets topics is rare.

However there is evidence of media influence on tweets, thus the existence of \newsreport topic in tweets. Meanwhile tweets out of \stlouis tend to be more influenced by news, so the difference between tweets in and out of \stlouis shows how media influence is. We find major topic \racism is consistent with increasing discussion of race in news. And there are relatively smaller changes of proportion of topics in tweets out of \stlouis, which could be influenced by diverse sources, such as social media, TV and news. The emergence of social media brings diverse opinions to the public. Although under the influence of news media, the public do not show similar focus as the main voice from media. Rather, they share similar emotions and focuses with people in \stlouis who experience and witness what happens. Social media enhances communication between the public and diversifies perspectives of events.
