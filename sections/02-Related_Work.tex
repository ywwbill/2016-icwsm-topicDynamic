\section{Related Work}
\label{sec:related_work}

We aim to study the interaction between media and the public by studying dynamics of news and tweets topics during the Ferguson unrest event. The most related research topics are agenda setting online. Political social scientists have noticed the change brought by Internet.~\newcite{sayre2010agenda} propose the question of roles and agenda setting form in the digital age. They manually analyze thousands of videos and news media on Proposition 8 in California, and find that post content in open social media reflects mainstream news, while posts also have influence on professional media coverage, thus create opinion dynamics. The results innovatively reveal the relationship between media and the public in a descriptive way. However it needs large amount of human efforts to apply this method to other events, and it is hard to identify the weight change of topics during the evolving process.

Topic detection has been widely studied for news and social media.
Tracking of memes on the Internet is a way to understand dissemination of certain information content.
Memes are entities that represent units of information at the desired level of detail~\cite{ratkiewicz2010detecting}.
The semantic units play as clear clues for detecting dynamic change of diverse topics.~\newcite{leskovec2009meme} introduce a meme-tracking technique and use it to track topic shifts in news and blogs over a long time.
They observe a 2.5-hour lag between peaks of attention to a phrase in the news media and in blogs, which shows evidence of possible interaction between media and other individuals.
It involves tracking of repeating ``semantic entities", as a way for topic tracking, but according to the algorithm only repeated topics, thus memes, can be detected.
Entity extraction is also a way to track information diffusion on the Internet.~\newcite{kim2012event} extract 5W1H (\emph{Who}, \emph{What}, \emph{Where}, \emph{When}, \emph{Why} and \emph{How}) structure from events and detect the connection between news, blogs and social networking in a supervised way.
However entity expression is limited to specific events rather than topics or frames.

A lot of work is to understand public opinions in social events such as the 2011 Tunisian and Egyptian Revolutions~\cite{gonzalez2011dynamics}, information sharing in protest during the G20 meetings in 2009~\cite{earl2013protest}, and organization of movement in the Occupy Wall Street protest~\cite{conover2013geospatial}.
These studies focus more on information flow between different roles\psrcomment{?} rather than the evolving of topics.
Topic detection in tweets has been studied because there is challenge for the short length of texts~\cite{yan2013biterm,zhao2011comparing}.

Aggregating short tweets into long document units is a way to alleviate the problem, such as author-based aggregation which is used to identify user topics and measure user similarity~\cite{weng2010twitterrank}. Aggregation based on time slices is used to track emerging events in Twitter~\cite{lau2012line}.~\newcite{hong2010empirical} also propose a method to aggregate tweets by similar words, which tend to work better than regular LDA. However the performance of these methods depends much on the research question and available datasets.

Other variations of topic model are proposed to discover topics from tweets. Biterm topic model~\cite[BTM]{yan2013biterm} directly simulates the generation of word co-occurrence patterns in corpus, thus enhances learning and leads to more coherent topics. A similar model is Word Network Topic Model~\cite{zuo2014word}, which also utilizes word co-occurrence network to solve the sparsity problem.~\newcite{zhao2011comparing} propose a Twitter-LDA model, which assumes that a single tweet is usually about a single topic, to discover topics from a representative sample of the entire Twitter. Supervised methods have also been applied to find tweets topics, such as Labeled-LDA~\cite{ramage2009labeled}, which maps the content of the tweets into dimensions of substance, style, status and social characteristics~\cite{ramage2010characterizing}.

There are studies working on alignment of events information and tweets, which are similar to our work.~\newcite{hu2012lda} propose a joint Bayesian model for events transcripts and tweets, which supposes that event information can impost topical influence on tweets.~\newcite{gao2012joint} create a joint topic model to extract important and complementary pieces of information across news and tweets, and generate complimentary information from both.
These models infer tweets topics incorporating information from event transcripts or news.
So tweets and event or news topics are under a unified frame.\psrcomment{Do you compare against other joint news/tweet models?}\wycomment{Currently no ...}
The difference is that, in our method, each tweet is treated to have one topic instead of a probability of distribution of topics.
We bring news and tweets together for training, and prove that \stlda model performs better for discovery of topic dynamics.
