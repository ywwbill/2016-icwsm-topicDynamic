\section{Related Work}
\label{sec:related_work}

\subsection{Interaction between Social Media and News}
\label{subsec:interaction}
% influence of media to public opinions.
% interaction between social media and news
We aim to study the interaction between media and the public which is closely related with agenda setting online. Political social scientists have noticed the change brought by Internet.~\newcite{sayre2010agenda} propose the question of roles and agenda setting form in the digital age. They manually analyze thousands of videos and news media on Proposition 8 in California, and find that post content in open social media reflects mainstream news, while posts also have influence on professional media coverage, thus create opinion dynamics. The results innovatively reveal the relationship between media and the public in a descriptive way. However it needs large amount of human efforts to apply this method to other events, and it is hard to identify the weight change of topics during the evolving process.~\newcite{hua2016topical} explore the semantic and topical relationships between news and social media data to reveal topic influences among multiple datasets. Their work is the first attempt to study how news influence social media data, however they focus the influence between topics based on word probability, ignoring time element of tweets.~\newcite{leskovec2009meme} introduce a meme-tracking technique and use it to track topic shifts in news and blogs over a long time. They observe a 2.5-hour lag between peaks of attention to a phrase in the news media and in blogs, which shows evidence of possible interaction between media and other individuals.

Social media content, especial Tweets have been studied as newswire for news~\cite{macdonald2013can,subavsic2011peddling,sankaranarayanan2009twitterstand}. However, their focus is on usage of valuable self-reporting tweets as distributed news wires, thus enrich the existing news system. So tweets that are about opinions and emotions may be removed as noise. In our study, we use all tweets as a whole to represent opinions from the public, and compare the opinion change of the public with that of news.
Some work pays attention to news sharing on social network, for example~\newcite{lee2012news} explore personal motivations influencing news sharing in social media;~\newcite{bandari2012pulse} predict the popularity of news on social media based on news features such as category, subjectivity, and source score, while we aim to find the extrinsic influence of media on a group of people based on topic changes, which is macro-scope analysis.

\subsection{Topic Dynamics}
\label{subsec:topic}
% Topic detection in social media to find public opinions
% Other topic dynamics

Social media have been studied to understand public opinions in social events such as the 2011 Tunisian and Egyptian Revolutions~\cite{gonzalez2011dynamics}, information sharing in protest during the G20 meetings in 2009~\cite{earl2013protest}, and organization of movement in the Occupy Wall Street protest~\cite{conover2013geospatial}. They all focus on the information flow inside the social media system, thus information spread between different users. However in our study, we study news as source of information and see whether there is influence of news on tweets.

Topic dynamics have also been studied a lot in understanding development of scientific areas~\cite{mane2004mapping}, burst topics in publications~\cite{he2010topic} and analysis of reviews from social media~\cite{lin2012tracking}.~\newcite{morinaga2004tracking} employ a finite mixture model to recognize the emergency, growth, and decay of each topic in a system. All the studies involve with single source of data and define calculation of topic dynamics in different ways. In our study, we bring tweets and news under a unified topic frame so that we can compare topic dynamics of news and tweets. Topic dynamic is defined as shift of topic proportions in daily window.


\subsection{Detection of Topics in News and Social Media}
\label{subsec:detection}
% techniques to find topics in tweets and the strength of our method
Comparison of topic dynamics involves detection of topics in news and tweets. Tracking of memes on the Internet is a way to understand evolvement of information content online. Memes are entities that represent units of information at the desired level of detail~\cite{ratkiewicz2010detecting}. The semantic units play as clear clues for detecting dynamic change of diverse topics, however according to the algorithm only repeated topics, thus memes, can be detected.

Topic detection in tweets has been a challenge because of the short length of texts~\cite{yan2013biterm,zhao2011comparing}. Aggregating short tweets into long document units is a way to alleviate the problem, such as author-based aggregation which is used to identify user topics and measure user similarity~\cite{weng2010twitterrank}. Aggregation based on time slices is used to track emerging events in Twitter~\cite{lau2012line}.~\newcite{hong2010empirical} also propose a method to aggregate tweets by similar words, which tend to work better than regular LDA. Biterm topic model~\cite[BTM]{yan2013biterm} directly simulates the generation of word co-occurrence patterns in corpus, thus enhances learning and leads to more coherent topics. A similar model is Word Network Topic Model~\cite{zuo2014word}, which also utilizes word co-occurrence network to solve the sparsity problem. However the performance of these methods depends much on the research question and available datasets. To train news and tweets together,~\newcite{hu2012lda} propose a joint Bayesian model for events transcripts and tweets, which supposes that event information can impost topical influence on tweets.~\newcite{gao2012joint} create a joint topic model to extract important and complementary pieces of information across news and tweets, and generate complimentary information from both. However in these models tweets are still a distribution of topics. We propose a \stlda model which assigns only one topic to each tweet and trains tweets and news together. Experiment results show that the topics assigned to tweets by \stlda are of better quality and perform better in topic dynamic discovery. 