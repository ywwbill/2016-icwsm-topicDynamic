\section{Introduction}
\label{sec:intro}

Cascading activation model is a widely accepted model which explores the relationship between government, media and the public~\cite{entman1993framing}. The model helps to explain how frame of information extends down from the White House to elites, media and then to the public. The government and elites enjoy more freedom for decisions because of power and independent environment. Information moves downward along the cascade with framing of upper layers and becomes limited to highlights. Although it acknowledges variation and the possible ways that news feeds back information about the public to influence actions of higher levels, the stair structure emphasizes heavily on the influence from media to the public. However the emergence of social networks seems to have changed the situation.

The creation of social networks aims to provide everyone with equal access to the world, to encourage more independent role in expression, and to achieve information democracy. It leads to increasing participation of information spread, opinion expression and activism of the public~\cite{gonzalez2011dynamics,tufekci2012social}. Social networks become a new competition battlefield for different frames from the media, elites, and the majority part of the public. Influence from the public may gain power because their voice are to be widely heard, and they are even able to participate and organize activities online. During the Arab Spring, Twitter has promoted the protest mobilization through reporting real-time event and provided basis for collaboration and emotional mobilization~\cite{breuer2014social}.

It leads us to rethink about the cascade model, specifically on the influence from media to the public. As a special function role in society, conventional media such as news and TV are still distinctive, but information spreading direction seems no longer heavily biased from media to the public. To figure out what changes have been brought by social networks, we study the following questions:

\begin{enumerate}
\item Is the public still heavily influenced by frames of the conventional media?
\item Does the media pay attention to what the public concerns and reflect the voice from the public?
\end{enumerate}

The formation of public opinions are complicated. It is related to multiple factors other than influence of the mass media, such as different perception of events due to distance~\cite{he2015uncovering}, demography and race ~\cite{page1987moves} etc. Meanwhile,~\newcite{kwak2010twitter} and~\newcite{ning2015uncovering} show that as a self-contained system, Twitter users are also heavily influenced by other tweets. So there are challenges to examine media influence on tweets considering different personal situations, the influence from within the system. It is hard to get detailed personal information and relate it with opinion change. However we could firstly exclude the distance factor according to geo-location of the public.Meanwhile, we detect short-term influence here, because the long-term influence of the media is hard to detect in a limited time window.

In this study, we take Ferguson unrest event in 2014 as an example, and analyze news and tweets topics along with the evolvement of the case. News reflect the voice from media while tweets reflect public opinions. To make topics of news and tweets comparable, we propose a Single Topic LDA (\stlda) model to bring news and tweets under a unified frame, while every tweet has only one topic but news are a distribution of multiple topics. We explore the research questions by examining shift of focus in news and tweets, and what may lead to the dynamic of topics, specifically on the possible relation to burst, emergence and decay of certain topics, thus the interaction between media and the public. Assuming the social media influence is similar for all people, but people out of \stlouis tend to be influenced by media more because they can't experience and report as witness. So the difference between tweets in and out of \stlouis could reflect media influence.

The contribution of this study lays in three main aspects. First, we solve the technical problem of building topic model for a mixture of short and long documents. Conventional topic models such as LDA and PLSA~\cite{hofmann1999probabilistic} perform badly because co-occurrence patterns in short texts are sparse. Our model considers the words in a tweet as a whole and assigns only one topic to a tweet, so that the main topic is more likely to be assigned.
%Consider a tweet with 5 words, every word's topic assignment makes up 20\% of its topic proportion, but usually a tweet's main topic is decided by 2 or 3 words. In this case, the main topic proportion is about 40\% to 60\%, which does not reflect the truth.
Second, we bring the cascade activation model under the environment with social media, re-examine the influence of different roles in the model, and bring new understanding of role of the media and formation of public opinions in social media.
Third, this preliminary study offers a method to identify strong frames, which has been a challenge for social scientists. A frame is an organization of reality by providing meaning to unfolding strip of events and promote particular definitions and interpretations of issues~\cite{chong2007framing}. By examining the topic dynamics between news and social media, which are separately representative of media and the public, we may understand which topics in news are influential and popular in the public, thus which frames are more competitive.
